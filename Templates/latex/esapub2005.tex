%%
%% This is file `esapub.tex',
%% generated with the docstrip utility.
%%
%% The original source files were:
%%
%% esapub.dtx  (with options: `manual')
%% ============================================
%% This is the manual describing the usage of
%%      esapub.cls
%% ============================================
%% Copyright 1999 Patrick W Daly
%% Max-Planck-Institut f\"ur Aeronomie
%% Max-Planck-Str. 2
%% D-37191 Katlenburg-Lindau
%% Germany
%% E-mail: daly@linmpi.mpg.de
%%
%% -------------------------------------------------
\ProvidesFile{esapub.tex}
          [2001/04/25 1.1 (PWD)]
\documentclass[a4paper,twocolumn]{esapub2005} % European paper
\pagestyle{empty}

% introduce this option for the ESA publications style
\bibliographystyle{alpha}

\usepackage{times}
\usepackage{natbib}

\title{A \LaTeX\ Class File for ESA Proceedings}
\author{Patrick W. Daly}
\author{A. N. Other}
\affil{Max-Planck-Institut f\"ur Aeronomie, 37191 Katlenburg-Lindau, Germany}
\author{F. Favata}
\affil{Astrophysics Division, ESTEC, 2200 AG Noordwijk, The Netherlands}

\newcommand{\btx}{\textsc{Bib}\TeX}
\newcommand{\filename}{esapub}

\begin{document}

\keywords{\LaTeX; ESA; macros}

\maketitle

\begin{abstract}
Talks presented at an ESA sponsored conference are published by the ESA
Publications Division by means of author-produced camera-ready copy. The
format is predetermined. This \LaTeX\ class allows authors to produce
this format with a straight-forward \LaTeX\ file. The only non-standard
features are the \verb!\keywords! command and the method for entering
author and affiliation names. \btx\ support is provided as a bibliography
style file for the publicly available package \texttt{natbib}.
\end{abstract}

\section{Introduction}

This is a \LaTeXe\ class based on the standard \texttt{article} class for
generating camera-ready copy of conference proceedings for publication by
the European Space Agency.

It conforms to the specifications for ESA conference proceedings:
\begin{itemize}
\item 23.5\,cm high and 17\,cm wide;
\item if two columns, they have 1\,cm between them;
\item 10\,pt font on 11\,pt, Times Roman preferred;
\item all titles in upper case
\end{itemize}

Compatibility with standard \LaTeX\ is maintained as much as possible in
order to simplify the transfer of text from or to another format. The
only additional features are the \verb!\keyword! command and the entry of
authors and affiliations.

It replaces the \LaTeX~2.09 style file that has been previously provided
by the ESA.

For an excellent manual on using \LaTeX, see Kopka \& Daly, \emph{A Guide
to \LaTeX}, 3rd ed., 1999, Addison Wesley Longman.

\section{Invoking the Class}

The class file is invoked with the \verb!\documentclass! command, as

\verb!\documentclass[a4paper,twocolumn]{!\texttt{\filename}\verb!}!

where the option \texttt{a4paper} may be replaced by \texttt{letterpaper}
(the default) for American installations. The text will be centered on the
specified paper type. The \texttt{twocolumn} option is given if the
publication is to be in two columns per page.

\subsection{Other Packages}

There are other (nearly) standard packages that may be included with
the \verb!\usepackage! command:
\begin{itemize}
\item \texttt{times} to use TimesRoman instead of Computer Modern (\TeX)
   fonts ,
\item \texttt{graphicx} or \texttt{epsfig} for importing figures (see
   Section~\ref{sec:grf}),
\item \texttt{natbib} to use author--year citations with \btx\ (see
    Section~\ref{sec:refs}).
\end{itemize}

The author may have his or her own extra packages, such as
\texttt{amsmath} for advanced mathematical formatting.

\section{Non-standard Features}

A list of key words is to be printed below the abstract. They are entered
\emph{anywhere before the abstract environment} with the \verb!\keywords!
command.
\begin{verbatim}
\keywords{space; plasmas; electrons}
\begin{abstract}
. . .
\end{abstract}
\end{verbatim}

Each author name should be entered with an \verb!\author! command. Give
the affiliation with the \verb!\affil! command after all authors of the
same affiliation. They will then be listed with a common footnote number.
\begin{verbatim}
\author{Donald P. Duck}
\author{Mickey C. Mouse}
\affil{Disney Studios, CA USA}
\author{Bugs G. Bunny}
\affil{Warner Bros. Studios}
\end{verbatim}
produces
\begin{center}
Donald P. Duck\textsuperscript{1}, Mickey C. Mouse\textsuperscript{1},
Bugs G. Bunny\textsuperscript{2}\\[0.5ex]
\textsuperscript{1}Disney Studios, CA USA\\
\textsuperscript{2}Warner Bros. Studios
\end{center}

For more control over the footnote numbers, one can give them explicitly
as optional arguments to \verb!\author! and \verb!\affil!:
\begin{verbatim}
\author[*]{Donald P. Duck}
\author[\dag]{Bugs G. Bunny}
\author[*]{Mickey C. Mouse}
\affil[*]{Disney Studios, CA USA}
\affil[\dag]{Warner Bros. Studios}
\end{verbatim}
to obtain
\begin{center}
Donald P. Duck\textsuperscript{*},
Bugs G. Bunny\textsuperscript{\dag},
Mickey C. Mouse\textsuperscript{*}\\[0.5ex]
\textsuperscript{*}Disney Studios, CA USA\\
\textsuperscript{\dag}Warner Bros. Studios
\end{center}


\section{Structure of the Document}

Except for the above features, the \texttt{\filename} class is identical to
the standard \texttt{article} class, as far as input is concerned. The
document should be organized as usual.

\noindent
\begin{small}
\itshape
\verb!\documentclass[a4paper,twocolumn]{!\texttt{\upshape\filename}\verb!}!\\
\% Any extra packages\\
\verb!\usepackage{times,natbib,graphicx,...}!\\
\% Title and authors\\
\verb!\title{!Title text\verb!}!\\
\verb!\author{!First Author\verb!}!\dots\\
\verb!\affil{!First affiliation\verb!}!\\
\dots\\
\% Start of body\\
\verb!\begin{document}!\\
\verb!\maketitle!\\
\% Keywords and abstract\\
\verb!\keywords{!keyword1; keyword2; \dots\verb!}!\\
\verb!\begin{abstract}!\\
Text of abstract\\
\verb!\end{abstract}!\\
\% Main text\\
\verb!\section{!Heading\verb!}!\\
Text\\
\verb!\subsection{!Sub-heading\verb!}!\\
Text\\
\verb!\section*{Acknowledgments}!\\
Acknowledgment text\\
\% Bibliography (Section~\ref{sec:refs})\\
\verb!\bibliographystyle{aa}!\\
\verb!\bibliography{!database name\verb!}!\\
\% Termination\\
\verb!\end{document}!
\end{small}

\begin{figure*}
\centering
\vspace{4cm}
\caption{Sample figure showing how an encapsulated PostScript graphic may
be included. This example is for a double column figure, which does cause
more placement problems than single column ones.\label{fig:double}}
\end{figure*}

\section{Bibliographic References}

Citations to bibliographic references are to be of the author--year
style. This may be done manually or with the help of
\texttt{natbib} package and \btx\ with the supplied \texttt{aa.bst}
bibliographic style file.

\subsection{Manual Citations}

The citations are either parenthetical \citep{smith96} or in-text
as shown by \citet{allen73} and elaborated on by \citet{nobody97}.

The list of references is placed at the end of the article, as
\begin{small}
\begin{verbatim}
\begin{thebibliography}{}
\bibitem{}
  Allen C., 1973, Astrophysical Quantities,
  Athlone Press
\bibitem{}
  Nobody B., Somebody G., Who M.E., et~al.,
  1997, ApJ 331, 902
\bibitem{}
  Smith A., Jones B., 1996, A\&A 555, 999
\end{thebibliography}
\end{verbatim}
\end{small}
Note the empty braces after \verb!\bibitem! and after
\verb!\begin{thebibliography}!

The format of the reference list is that used by \emph{Astronomy and
Astrophysics} and other astronomy journals.

\subsection{References with \texttt{natbib}}\label{sec:refs}

\renewcommand{\/}{\discretionary{/}{}{/}}
The \texttt{natbib} package is a powerful tool for extending the citation
and bibliography features of standard \LaTeX. It is included in most
modern installations these days.%
\footnote{Otherwise it may be obtained from
\texttt{ftp:\//ctan.tug.org\/tex-archive\/macros\/latex\/contrib\/supported\/natbib}
or from the other CTAN servers or from almost any CD-ROM distribution of
\TeX/\LaTeX.}

With \texttt{natbib}, in-text citations are generated with
\verb!\citet{allen73}! to yield ``\citet{allen73}'' while parenthetical
ones are made with \verb!\citep{smith96}! for \citep{smith96}. There are
many other possibilities, such as \verb!\citeauthor! for the authors
without year. See the \texttt{natbib} documentation.

With \texttt{natbib} the bibliography must be entered differently, at
least the \verb!\bibitem! entries.
\begin{small}
\begin{verbatim}
\bibitem[Allen(1973)]{allen73}
. . .
\bibitem[Nobody et~al.(1997)]{nobody97}
. . .
\bibitem[Smith \& Jones(1996)]{smith96}
. . .
\end{verbatim}
\end{small}
The text in square brackets contains the author and year information,
with the year part in parentheses, no space before, which is used by the
\verb!\citet! and \verb!\citep! commands.

Rather than trying to make up the \texttt{thebibliography} environment
manually with all its intricate details, one can let \btx\ do it, if the
references are already in an appropriate database file. For this
purpose, a bibliographic style file \texttt{aa.bst} is provided with
\texttt{\filename.cls}, designed to produce output formatted for
\emph{Astronomy and Astrophysics} and the \texttt{natbib} package. In
this case, one replaces the \texttt{thebibliography} environment with
\begin{flushleft}
\verb!\bibliographystyle{aa}!\\
\verb!\bibliography{!\emph{bib file names}\verb!}!
\end{flushleft}
One processes the \LaTeX\ file once, then \btx, and then the \LaTeX\ file
at least twice. This need only be repeated if the citations have added
or deleted in the text.

\section{Figures and Tables}

Figures and tables are inserted with the normal \LaTeX\ environments
\texttt{figure} and \texttt{table}. They are numbered automatically and
one refers to the numbers with the \verb!\label! and \verb!\ref! system.

\subsection{Figures}\label{sec:grf}

\begin{figure}
\centering
\vspace{4cm}
\caption{Sample figure showing how an encapsulated PostScript graphic may
be included. This example is for a single column figure.\label{fig:single}}
\end{figure}

The \texttt{figure} environment is used to enter a single column figure
such as Figure~\ref{fig:single}, while \texttt{figure*} is for double
column figures (Figure~\ref{fig:double}).

\begin{small}
\begin{verbatim}
\begin{figure}
\centering
\includegraphics[width=0.8\linewidth]{sample.eps}
\caption{Sample figure showing how an encapsulated
PostScript graphic may be included. This example
is for a single column figure.\label{fig:single}}
\end{figure}
\end{verbatim}
\end{small}
One can then refer to this figure with \verb!Figure~\ref{fig:single}!, producing
``Figure~\ref{fig:single}''.

The \verb!\includegraphics! command is made available with the
\texttt{graphicx} package and allows the importation of graphic files.
For PostScript output (with the \texttt{dvips} program) these graphics
must adhere to the \emph{encapsulated} PostScript standard.

Many users are familiar with the \verb!\epsfig! available with the
\texttt{epsfig} package. With this the syntax is slightly different:
\begin{small}
\begin{verbatim}
\epsfig{file=sample.eps,width=0.8\linewidth}
\end{verbatim}
\end{small}
(In fact, the \texttt{epsfig} package uses the \texttt{graphicx} package
so in the end they do exactly the same thing.)

The same syntax can also be used with pdf\TeX, a variant on the \TeX\
program producing PDF output directly. In this case, the figures must be
in PDF, PNG, or JPEG format. It is not necessary to include the extension
in the file name (\texttt{file=sample} suffices), something that makes
the \LaTeX\ text more general for both normal \TeX\ and pdf\TeX. (It may
however be necessary to add the option \texttt{[pdftex]} when loading the
graphics packages.)

\subsection{Tables}
\begin{table}
  \begin{center}
    \caption{A sample table illustrating usage of the \LaTeX{} table
      environment.}\vspace{1em}
    \renewcommand{\arraystretch}{1.2}
    \begin{tabular}[h]{lrcc}
      \hline
      First column & Col. 2      &  Col. 3      &  V mag \\
      \hline
      row 1  & 11.0 & 25.0 & 12 \\
      row 2  & 11.0 & 25.0 & 12 \\
      row 3  & 11.0 & 25.0 & 12 \\
      row 4  & 11.0 & 25.0 & 12 \\
      row 5  & 11.0 & 25.0 & 12 \\
      \hline \\
      \end{tabular}
    \label{tab:table}
  \end{center}
\end{table}

Tables are placed and numbered and referred to with the \texttt{table}
and \texttt{table*} environments. The contents of the table are normally
entered with the \texttt{tabular} or \texttt{tabbing} environments. The
\verb!\caption! now comes at the top of the table, before the table
contents.

\section{Equations}
Formulae which appear in the running text should be enclosed in
\texttt{\$} signs. For example, to produce the equation $a^2 + b^2 = c^2$
within a paragraph type \verb!$a^2 + b^2 = c^2$!. Displayed formulae are
produced using the \verb!\begin{equation}! and \verb!\end{equation}!
commands (see Equation~\ref{eq:equation1}). This produces equations which
are automatically numbered sequentially throughout your paper. Equations
which should appear together can be formatted using
\verb!\begin{eqnarray}! and \verb!\end{eqnarray}! as for
Equations~\ref{eq:equation2} and \ref{eq:equation3}:
\begin{equation}
 \Delta{\hat a}_i = \sum_j {\partial f_i \over \partial a_j}
\Delta a_j
\label{eq:equation1}
\end{equation}
\begin{eqnarray}
\alpha &= &\alpha_0 + (T-T_0)\,\mu_{\alpha *0}\,{\rm sec}\,\delta_0
\label{eq:equation2}\\
\delta &= &\delta_0 + (T-T_0)\,\mu_{\delta 0} \label{eq:equation3}
\end{eqnarray}

When in math mode (i.e.\ within the \texttt{equation} or
\texttt{eqnarray} environment) all letters appear in italics. However,
the preferred notation is for subscripts, superscripts\footnote{Except
when the superscript or subscript are variables.} and text within the
equation to be typeset as roman. To achieve this use the
\verb!{\mbox{..}}! command. Thus,
\verb!$T_{\mbox{eff}}=5.8\times10^{3}$~K! produces
$T_{\mbox{eff}}=5.8\times10^{3}$~K. Note that units should be tied to the
numerical value using \verb!~! and should always be in roman font (the
default outside of math mode).

\section{Final Manuscripts}
\subsection{Preparation of Final Manuscripts}
There is an upper limit of eight pages for an Invited paper, six pages
for Contributed papers and four pages for Poster papers.

You should use the standard \LaTeX{} command, in conjunction with the
style file provided, esapub.sty, to produce the final output.  If
using dvips places the final PostScript file lower on the page than
appears to be reasonable, try using the qualifier `-t a4' with dvips.

\subsection{Submission of Final Manuscripts}
Please refer to the conference guidelines for submission instructions.

\section*{Acknowledgments}

The section containing acknowledgments should use the \verb!\section*!
form, as shown, to prevent it from being numbered.

% The following bibliography was produced with
%   \bibliographystyle{aa}
%   \bibliography{esapub}
% The results are inserted directly here to simplify
% the demonstration.

\begin{thebibliography}{}
\bibitem[Allen(1973)]{allen73}
Allen C., 1973, Astrophysical Quantities, Athlone Press

\bibitem[Nobody et~al.(1997)]{nobody97}
Nobody B., Somebody G., Who M.E., et~al., 1997, ApJ 331, 902

\bibitem[Smith \& Jones(1996)]{smith96}
Smith A., Jones B., 1996, A\&A 555, 999

\end{thebibliography}
\end{document}
%%
%% <<<<< End of generated file <<<<<<
%%
%% End of file `esapub.tex'.
