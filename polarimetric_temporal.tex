\documentclass[11pt]{article}
%Gummi|063|=)
\title{\textbf{A Review On Temporal PolSAR/PolinSAR data}}
\author{Simone Baffelli}
\date{}


\usepackage{amsmath}
\usepackage{booktabs}
\usepackage{multirow}
\usepackage{array} 
\usepackage{fullpage}
\usepackage{rotating}
\usepackage[section]{placeins}
\usepackage{cite}
\usepackage{longtable}
\usepackage{multirow}
%%%%%%%%%%%Commands
\newcolumntype{L}[1]{>{\raggedright\let\newline\\\arraybackslash\hspace{0pt}}m{#1}}
\newcolumntype{C}[1]{>{\centering\let\newline\\\arraybackslash\hspace{0pt}}m{#1}}
\newcolumntype{R}[1]{>{\raggedleft\let\newline\\\arraybackslash\hspace{0pt}}m{#1}}

\begin{document}
	\maketitle
\begin{longtable}{L{1cm}L{5cm}L{3cm}L{3cm}l}
	Ref. & Short Description & Advantages & Disadvantages\\
	\hline
	 &
	Standard noise filtering (Boxcar, Median) &
	Very simple and relatively fast &
	Cannot be applied to multiplicative noise. Resolution loss.\\
	\hline\\
	\cite{Lee1980}& 
	MMSE Filter. Obtained assuming the filter for the pixels to be a linear combination of the prior mean of the reflectance and of the pixel intensity. Uses local statistics&
	Accounts for noise model. Adaptive, preserves edges.&
	Does not filter noise near edges.\\
	\hline\\
	\cite{Lee1981}&
	Similar to the MMSE filter. Select homogenous area for the statistics using edge aligned windows.&
	Better filtering: the selected pixels have similar radiometric properties.&
	A bit more complex than MMSE: edge selection is necessary.\\
	\hline\\
	\cite{Deledalle2010}&
	Nonlocal filtering approach: use the information contained in (possibly all) pixels basing on the similarity to the pixel to be filtered. The paper gives the expression for the weight in the case of SLC and interferometric images.&
	Preserves structures and resolution, efficent reduction of noise.&
	High variance in regions with few redundant patches.
\end{longtable}

\bibliographystyle{IEEEtran}
\bibliography{/home/baffelli/Literature/literature.bib}{}

\end{document}